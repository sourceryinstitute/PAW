\documentclass{article}

\title{Computer Science Challenges to Imaging the Universe with the SKA Radio-Telescope}
\author{Peter J. Braam\thanks{Physics Department, University of Oxford}}
\date{November 17, 2019}
\providecommand{\keywords}[1]{\textbf{\textit{Keywords---}} #1}
\begin{document}

\maketitle

% Here is the abstract.
\begin{abstract}
	The SKA radio telescope will be a massive scientific instrument entering service in the late 2020s. The conversion of its antenna signals to images and the detection of transient phenomena is a truly massive computational undertaking, requiring 200PB/sec of memory bandwidth, involving domain-specific data. In this lecture, we will give an overview of the data processing in the telescope and the process that has been followed to design suitable algorithms and systems. We will highlight the difficulties observed in utilizing existing programming language approaches and discuss opportunities for innovation from the computer science perspective. 
\end{abstract}

\keywords{Hardware, Emerging Technologies, Memory and Dense Storage
Software and its engineering, Software Organization and Properties, Software System Structures, Ultra-large-scale Systems
Programming Languages}
\end{document}

